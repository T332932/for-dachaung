\documentclass[12pt,a4paper]{article}
\usepackage{ctex}
\usepackage{amsmath,amssymb}
\usepackage{geometry}
\usepackage{graphicx}
\usepackage{tikz}
\usetikzlibrary{arrows.meta,patterns,calc}
\geometry{left=2cm,right=2cm,top=2.5cm,bottom=2.5cm}
\begin{document}
10. 下图是函数 $y = \sin(\omega x + \phi)$ 的部分图像, 则 $\sin(\omega x + \phi)=$

  ![函数图像](...)

\par A. $\sin(x + \frac{\pi}{3})$

\par B. $\sin(\frac{\pi}{3} - 2x)$

\par C. $\cos(2x + \frac{\pi}{6})$

\par D. $\cos(\frac{5\pi}{6} - 2x)$


\par\noindent\hfill\begin{minipage}{0.45\textwidth}\centering
\begin{tikzpicture}[scale=0.75, line width=0.5pt, >=Stealth[length=4pt], every node/.style={font=\small, inner sep=1pt}]
\draw[->] (-0.5, 0) -- (3.5, 0) node[below] {$x$};
\draw[->] (0, -1.5) -- (0, 1.5) node[left] {$y$};
\node[below left] at (0,0) {$O$};

\draw[domain=-0.3:3.2, samples=100, smooth, thick, color=blue] plot (\x, {sin((\x r) + pi/3 r)});

\draw (1.57, 0.05) -- (1.57, -0.05) node[below] {$\frac{2\pi}{3}$};
\node at ({pi/6}, -0.05) {$\shortmid$};
\node[below] at ({pi/6}, -0.05) {$\frac{\pi}{6}$};
\draw (-0.05, 1) -- (0.05, 1) node[right] {$1$};

\draw[dashed] ({pi/6}, 0) -- ({pi/6}, 1) -- (0, 1);
\node[circle, fill, inner sep=1pt] at ({pi/6}, 1) {};
\node[circle, fill, inner sep=1pt] at ({2*pi/3}, 0) {};

% Corrected x-axis tick labels to decimal for pgfplots compatibility
% x=pi/6 is approx 0.52
% x=2pi/3 is approx 2.09
\draw[dashed] (0.52, 0) -- (0.52, 1) -- (0, 1);
\node[circle, fill, inner sep=1pt] at (0.52, 1) {};
\node[circle, fill, inner sep=1pt] at (2.09, 0) {};
\draw (2.09, 0.05) -- (2.09, -0.05) node[below] {$\frac{2\pi}{3}$};
\node at (0.52, -0.05) {$\shortmid$};
\node[below] at (0.52, -0.05) {$\frac{\pi}{6}$};

\end{tikzpicture}
\end{minipage}\hfill\null



\textbf{答案:} 由函数图像可知...(省略)...故选 A。\end{document}
