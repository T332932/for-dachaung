\documentclass[no-math]{ctexart}
\setCJKmainfont{Noto Serif CJK SC}
\setCJKsansfont{Noto Sans CJK SC}
\setCJKmonofont{Noto Sans Mono CJK SC}
\everymath{\displaystyle}

\usepackage{amsmath,amssymb}
\usepackage{tikz}
\usetikzlibrary{arrows.meta,patterns,calc}
\usepackage{graphicx}
\usepackage{enumitem}
\setenumerate{itemsep=0pt,partopsep=0pt,parsep=\parskip,topsep=0pt}

\usepackage[paperheight=26cm,paperwidth=18.4cm,left=2cm,right=2cm,top=1.5cm,bottom=2cm,headsep=10pt]{geometry}
\usepackage{fancyhdr}
\pagestyle{fancy}
\renewcommand{\headrulewidth}{0pt}
\usepackage{lastpage}
\usepackage[bodytextleadingratio=1.67,restoremathleading=true]{zhlineskip}
\usepackage{ifthen}

% 选项自适应排版命令
\newcommand{\onech}[4]{\makebox[3.4cm][l]{{\sf A}.#1}\makebox[3.4cm][l]{{\sf B}.#2}\makebox[3.4cm][l]{{\sf C}.#3}\makebox[3.4cm][l]{{\sf D}.#4}}
\newcommand{\twoch}[4]{\makebox[6.8cm][l]{{\sf A}.#1}\makebox[6.8cm][l]{{\sf B}.#2}\\ \makebox[6.8cm][l]{{\sf C}.#3}\makebox[6.8cm][l]{{\sf D}.#4}}
\newcommand{\fourch}[4]{{\sf A}.#1\\ {\sf B}.#2\\ {\sf C}.#3\\ {\sf D}.#4}

\newlength\widthcha
\newlength\widthchb
\newlength\widthch
\newlength\fourthtabwidth
\setlength\fourthtabwidth{0.22\textwidth}
\newlength\halftabwidth
\setlength\halftabwidth{0.45\textwidth}

\newcommand{\choice}[4]{%
  \settowidth\widthcha{{\sf A}M.#1}%
  \setlength{\widthch}{\widthcha}%
  \settowidth\widthchb{{\sf B}M.#2}%
  \ifthenelse{\lengthtest{\widthch < \widthchb}}{\setlength{\widthch}{\widthchb}}{}%
  \settowidth\widthchb{{\sf C}M.#3}%
  \ifthenelse{\lengthtest{\widthch < \widthchb}}{\setlength{\widthch}{\widthchb}}{}%
  \settowidth\widthchb{{\sf D}M.#4}%
  \ifthenelse{\lengthtest{\widthch < \widthchb}}{\setlength{\widthch}{\widthchb}}{}%
  \ifthenelse{\lengthtest{\widthch < \fourthtabwidth}}{\onech{#1}{#2}{#3}{#4}}%
  {\ifthenelse{\lengthtest{\widthch < \halftabwidth}}{\twoch{#1}{#2}{#3}{#4}}%
  {\fourch{#1}{#2}{#3}{#4}}}%
}

% 填空横线(兼容数学模式和文本模式)
\newcommand{\undsp}{\underline{\makebox[3em]{}}}

\begin{document}
\SetMathEnvironmentSinglespace{1}
\lineskiplimit=5.5pt
\lineskip=7pt
\abovedisplayshortskip=5pt
\belowdisplayshortskip=5pt
\abovedisplayskip=5pt
\belowdisplayskip=5pt

\fancyfoot[C]{\bf\sf 数学试题 第{\sf\thepage} 页 (共~{\sf\pageref{LastPage}}~页)}

\begin{center}
\zihao{2}\heiti 2024新高考1卷
\end{center}
\begin{enumerate}[align=left,labelindent=0em,labelwidth=2em,labelsep=0em,leftmargin=2em]
\item[{\bf 一、}]{\bf\sf 选择题:本题共 11 小题,每小题 5 分,共 58 分。在每小题给出的四个选项中,只有一项是符合题目要求的。}
\end{enumerate}
\begin{enumerate}[align=left,labelindent=0em,label={\bf\sf\arabic*.},labelwidth=1.5em,labelsep=0em,leftmargin=1.5em,itemsep=0pt,topsep=0pt,start=1]
\item 已知集合 $A = \{x | -5 < x^3 < 5\}, B = \{-3, -1, 0, 2, 3\}$,则 $A \cap B$ ( )
\\
\choice{$\{-1, 0\}$}{$\{2, 3\}$}{$\{-3, -1, 0\}$}{$\{-1, 0, 2\}$}
\item 若 $\frac{z}{z-1} = 1+i$,则 $z=$ ( )
\\
\choice{$-1-i$}{$-1+i$}{$1-i$}{$1+i$}
\item 已知向量 $\boldsymbol{a}=(0,1), \boldsymbol{b}=(2,x)$,若 $\boldsymbol{b} \perp (\boldsymbol{b}-4\boldsymbol{a})$,则 $x=$
\\
\choice{-2}{-1}{1}{2}
\item 已知 $\cos(\alpha + \beta) = m, \tan \alpha \tan \beta = 2$,则 $\cos(\alpha - \beta) = (\quad)$
\\
\choice{$-3m$}{$-\frac{m}{3}$}{$\frac{m}{3}$}{$3m$}
\item 已知圆柱和圆锥的底面半径相等,侧面积相等,且它们的高均为 $\sqrt{3}$,则圆锥的体积为( )
\\
\choice{$2\sqrt{3}\pi$}{$3\sqrt{3}\pi$}{$6\sqrt{3}\pi$}{$9\sqrt{3}\pi$}

\par\noindent\hfill\begin{minipage}{0.48\textwidth}\centering
\begin{tikzpicture}[>=Stealth, scale=0.8, line width=0.5pt]
\draw (1.500,6.000) rectangle (4.500,3.000);
\draw[dashed] (3.000,6.000) -- (3.000,3.000);
\draw (3.000,3.000) -- (4.500,3.000);
\node at (3.150,4.350) {$h$};
\node at (3.600,2.460) {$r$};
\node at (2.400,1.500) {$圆柱$};
\draw (7.500,3.000) -- (10.500,3.000) -- (9.000,6.000) -- cycle;
\draw[dashed] (9.000,6.000) -- (9.000,3.000);
\draw (9.000,3.000) -- (10.500,3.000);
\node at (9.150,4.350) {$h$};
\node at (9.600,2.460) {$r$};
\node at (9.900,4.350) {$l$};
\node at (8.400,1.500) {$圆锥$};
\end{tikzpicture}
\end{minipage}\hfill\null

\item 已知函数为 $f(x) = \begin{cases} -x^2-2ax-a, & x<0 \\ e^x+\ln(x+1), & x \ge 0 \end{cases}$,在 $\mathbf{R}$ 上单调递增,则 $a$ 取值的范围是( )
\\
\choice{$(-\infty, 0]$}{$[-1, 0]$}{$[-1, 1]$}{$[0, +\infty)$}
\item 当 $x \in [0, 2\pi]$ 时,曲线 $y = \sin x$ 与 $y = 2\sin\left(3x - \frac{\pi}{6}\right)$ 的交点个数为(  )
\\
\choice{3}{4}{6}{8}

\par\noindent\hfill\begin{minipage}{0.48\textwidth}\centering
\begin{tikzpicture}[>=Stealth, scale=0.8, line width=0.5pt]
\draw (0.600,4.500) -- (11.400,4.500);
\draw (0.600,8.700) -- (0.600,0.300);
\node at (11.550,4.350) {$x$};
\node at (0.450,8.700) {$y$};
\draw (3.300,4.650) -- (3.300,4.350);
\node at (3.300,3.900) {$π/2$};
\draw (6.000,4.650) -- (6.000,4.350);
\node at (6.000,3.900) {$π$};
\draw (8.700,4.650) -- (8.700,4.350);
\node at (8.700,3.900) {$3π/2$};
\draw (11.400,4.650) -- (11.400,4.350);
\node at (11.400,3.900) {$2π$};
\node at (0.540,4.050) {$O$};
\draw (0.450,6.600) -- (0.750,6.600);
\node at (0.300,6.450) {$1$};
\draw (0.450,2.400) -- (0.750,2.400);
\node at (0.300,2.250) {$-1$};
\draw (0.450,8.700) -- (0.750,8.700);
\node at (0.300,8.550) {$2$};
\draw (0.450,0.300) -- (0.750,0.300);
\node at (0.300,0.150) {$-2$};
\draw (0.600,4.500) -- (0.660,4.709) -- (0.720,4.917) -- (0.780,5.123) -- (0.840,5.326) -- (0.900,5.526) -- (0.960,5.722) -- (1.020,5.912) -- (1.080,6.098) -- (1.140,6.277) -- (1.200,6.450) -- (1.260,6.615) -- (1.320,6.772) -- (1.380,6.919) -- (1.440,7.056) -- (1.500,7.183) -- (1.560,7.299) -- (1.620,7.403) -- (1.680,7.495) -- (1.740,7.575) -- (1.800,7.641) -- (1.860,7.694) -- (1.920,7.733) -- (1.980,7.759) -- (2.040,7.770) -- (2.100,7.767) -- (2.160,7.750) -- (2.220,7.719) -- (2.280,7.674) -- (2.340,7.616) -- (2.400,7.546) -- (2.460,7.463) -- (2.520,7.369) -- (2.580,7.264) -- (2.640,7.149) -- (2.700,7.025) -- (2.760,6.893) -- (2.820,6.753) -- (2.880,6.606) -- (2.940,6.454) -- (3.000,6.297) -- (3.060,6.136) -- (3.120,5.973) -- (3.180,5.806) -- (3.240,5.639) -- (3.300,5.472) -- (3.360,5.305) -- (3.420,5.140) -- (3.480,4.977) -- (3.540,4.818) -- (3.600,4.663) -- (3.660,4.513) -- (3.720,4.369) -- (3.780,4.231) -- (3.840,4.099) -- (3.900,3.975) -- (3.960,3.859) -- (4.020,3.751) -- (4.080,3.652) -- (4.140,3.562) -- (4.200,3.481) -- (4.260,3.409) -- (4.320,3.347) -- (4.380,3.295) -- (4.440,3.254) -- (4.500,3.222) -- (4.560,3.201) -- (4.620,3.190) -- (4.680,3.189) -- (4.740,3.198) -- (4.800,3.217) -- (4.860,3.246) -- (4.920,3.285) -- (4.980,3.332) -- (5.040,3.388) -- (5.100,3.453) -- (5.160,3.525) -- (5.220,3.604) -- (5.280,3.690) -- (5.340,3.782) -- (5.400,3.879) -- (5.460,3.982) -- (5.520,4.088) -- (5.580,4.199) -- (5.640,4.313) -- (5.700,4.429) -- (5.760,4.547) -- (5.820,4.666) -- (5.880,4.786) -- (5.940,4.905) -- (6.000,5.023) -- (6.060,5.140) -- (6.120,5.256) -- (6.180,5.368) -- (6.240,5.479) -- (6.300,5.585) -- (6.360,5.687) -- (6.420,5.785) -- (6.480,5.877) -- (6.540,5.964) -- (6.600,6.044) -- (6.660,6.118) -- (6.720,6.185) -- (6.780,6.245) -- (6.840,6.296) -- (6.900,6.340) -- (6.960,6.375) -- (7.020,6.401) -- (7.080,6.418) -- (7.140,6.426) -- (7.200,6.424) -- (7.260,6.412) -- (7.320,6.390) -- (7.380,6.358) -- (7.440,6.317) -- (7.500,6.266) -- (7.560,6.205) -- (7.620,6.136) -- (7.680,6.058) -- (7.740,5.972) -- (7.800,5.878) -- (7.860,5.776) -- (7.920,5.668) -- (7.980,5.553) -- (8.040,5.433) -- (8.100,5.308) -- (8.160,5.179) -- (8.220,5.046) -- (8.280,4.911) -- (8.340,4.774) -- (8.400,4.636) -- (8.460,4.498) -- (8.520,4.360) -- (8.580,4.223) -- (8.640,4.089) -- (8.700,3.957) -- (8.760,3.829) -- (8.820,3.706) -- (8.880,3.588) -- (8.940,3.476) -- (9.000,3.370) -- (9.060,3.272) -- (9.120,3.181) -- (9.180,3.099) -- (9.240,3.025) -- (9.300,2.961) -- (9.360,2.906) -- (9.420,2.861) -- (9.480,2.826) -- (9.540,2.801) -- (9.600,2.787) -- (9.660,2.784) -- (9.720,2.792) -- (9.780,2.809) -- (9.840,2.838) -- (9.900,2.877) -- (9.960,2.925) -- (10.020,2.983) -- (10.080,3.050) -- (10.140,3.124) -- (10.200,3.208) -- (10.260,3.298) -- (10.320,3.395) -- (10.380,3.498) -- (10.440,3.606) -- (10.500,3.718) -- (10.560,3.835) -- (10.620,3.954) -- (10.680,4.076) -- (10.740,4.199) -- (10.800,4.323) -- (10.860,4.447) -- (10.920,4.571) -- (10.980,4.693) -- (11.040,4.812) -- (11.100,4.929) -- (11.160,5.042) -- (11.220,5.150) -- (11.280,5.255) -- (11.340,5.353) -- (11.400,5.445);
\draw (0.600,2.400) -- (0.660,2.095) -- (0.720,1.811) -- (0.780,1.550) -- (0.840,1.315) -- (0.900,1.109) -- (0.960,0.935) -- (1.020,0.794) -- (1.080,0.688) -- (1.140,0.618) -- (1.200,0.585) -- (1.260,0.589) -- (1.320,0.628) -- (1.380,0.703) -- (1.440,0.810) -- (1.500,0.948) -- (1.560,1.114) -- (1.620,1.303) -- (1.680,1.512) -- (1.740,1.738) -- (1.800,1.975) -- (1.860,2.219) -- (1.920,2.466) -- (1.980,2.711) -- (2.040,2.951) -- (2.100,3.182) -- (2.160,3.400) -- (2.220,3.600) -- (2.280,3.781) -- (2.340,3.938) -- (2.400,4.069) -- (2.460,4.173) -- (2.520,4.246) -- (2.580,4.288) -- (2.640,4.298) -- (2.700,4.275) -- (2.760,4.220) -- (2.820,4.133) -- (2.880,4.017) -- (2.940,3.872) -- (3.000,3.703) -- (3.060,3.510) -- (3.120,3.298) -- (3.180,3.070) -- (3.240,2.829) -- (3.300,2.579) -- (3.360,2.323) -- (3.420,2.064) -- (3.480,1.806) -- (3.540,1.554) -- (3.600,1.310) -- (3.660,1.077) -- (3.720,0.859) -- (3.780,0.659) -- (3.840,0.480) -- (3.900,0.324) -- (3.960,0.194) -- (4.020,0.091) -- (4.080,0.018) -- (4.140,-0.024) -- (4.200,-0.034) -- (4.260,-0.010) -- (4.320,0.046) -- (4.380,0.135) -- (4.440,0.255) -- (4.500,0.403) -- (4.560,0.578) -- (4.620,0.776) -- (4.680,0.993) -- (4.740,1.226) -- (4.800,1.471) -- (4.860,1.724) -- (4.920,1.980) -- (4.980,2.236) -- (5.040,2.487) -- (5.100,2.730) -- (5.160,2.962) -- (5.220,3.180) -- (5.280,3.381) -- (5.340,3.562) -- (5.400,3.721) -- (5.460,3.856) -- (5.520,3.967) -- (5.580,4.052) -- (5.640,4.111) -- (5.700,4.142) -- (5.760,4.145) -- (5.820,4.120) -- (5.880,4.068) -- (5.940,3.989) -- (6.000,3.884) -- (6.060,3.756) -- (6.120,3.606) -- (6.180,3.437) -- (6.240,3.250) -- (6.300,3.049) -- (6.360,2.836) -- (6.420,2.389) -- (6.480,2.161) -- (6.540,1.934) -- (6.600,1.712) -- (6.660,1.499) -- (6.720,1.297) -- (6.780,1.109) -- (6.840,0.938) -- (6.900,0.788) -- (6.960,0.659) -- (7.020,0.555) -- (7.080,0.476) -- (7.140,0.425) -- (7.200,0.402) -- (7.260,0.408) -- (7.320,0.443) -- (7.380,0.506) -- (7.440,0.597) -- (7.500,0.713) -- (7.560,0.852) -- (7.620,1.012) -- (7.680,1.190) -- (7.740,1.382) -- (7.800,1.584) -- (7.860,1.794) -- (7.920,2.007) -- (7.980,2.220) -- (8.040,2.430) -- (8.100,2.635) -- (8.160,2.831) -- (8.220,3.016) -- (8.280,3.188) -- (8.340,3.345) -- (8.400,3.485) -- (8.460,3.607) -- (8.520,3.710) -- (8.580,3.793) -- (8.640,3.855) -- (8.700,3.896) -- (8.760,3.917) -- (8.820,3.916) -- (8.880,3.895) -- (8.940,3.855) -- (9.000,3.796) -- (9.060,3.719) -- (9.120,3.626) -- (9.180,3.517) -- (9.240,3.395) -- (9.300,3.262) -- (9.360,3.118) -- (9.420,2.966) -- (9.480,2.807) -- (9.540,2.644) -- (9.600,2.478) -- (9.660,2.312) -- (9.720,2.147) -- (9.780,1.985) -- (9.840,1.828) -- (9.900,1.680) -- (9.960,1.541) -- (10.020,1.414) -- (10.080,1.299) -- (10.140,1.198) -- (10.200,1.113) -- (10.260,1.045) -- (10.320,0.994) -- (10.380,0.962) -- (10.440,0.948) -- (10.500,0.954) -- (10.560,0.979) -- (10.620,1.022) -- (10.680,1.084) -- (10.740,1.163) -- (10.800,1.260) -- (10.860,1.371) -- (10.920,1.497) -- (10.980,1.636) -- (11.040,1.786) -- (11.100,1.945) -- (11.160,2.111) -- (11.220,2.283) -- (11.280,2.460) -- (11.340,2.638) -- (11.400,2.400);
\draw (8.400,8.400) -- (9.000,8.400);
\node at (9.300,8.250) {$y = sin(x)$};
\draw (8.400,7.800) -- (9.000,7.800);
\node at (9.300,7.650) {$y = 2sin(3x-π/6)$};
\end{tikzpicture}
\end{minipage}\hfill\null

\item 已知函数 $f(x)$ 的定义域为 $\mathbf{R}$,$f(x) > f(x-1) + f(x-2)$,且当 $x<3$ 时 $f(x)=x$,则下列结论中一定正确的是( )
\\
\choice{$f(10) > 100$}{$f(20) > 1000$}{$f(10) < 1000$}{$f(20) < 10000$}
\item 为了了解推动出口后的亩收入(单位:万元)情况,从该种植区抽取样本,得到推动出口后亩收入的样本均值 $\bar{x}=2.1$,样本方差 $s^2=0.01$,已知该种植区以往的亩收入 $X$ 服从正态分布 $N(1.8, 0.1^2)$,假设推动出口后的亩收入 $Y$ 服从正态分布 $N(\bar{x}, s^2)$,则( )(若随机变量 $Z$ 服从正态分布 $N(u, \sigma^2)$,$P(Z < u+\sigma) \approx 0.8413$)
\\
\choice{$P(X > 2) > 0.2$}{$P(X > 2) < 0.5$}{$P(Y > 2) > 0.5$}{$P(Y > 2) < 0.8$}
\item 设函数 $f(x)=(x-1)^2(x-4)$,则 ( )
\\
\choice{$x=3$ 是 $f(x)$ 的极小值点}{当 $0<x<1$ 时, $f(x)<f(x^2)$}{当 $1<x<2$ 时, $-4<f(2x-1)<0$}{当 $-1<x<0$ 时, $f(2-x)>f(x)$}

\par\noindent\hfill\begin{minipage}{0.48\textwidth}\centering
\begin{tikzpicture}[>=Stealth, scale=0.8, line width=0.5pt]
\draw (0.600,4.500) -- (11.400,4.500);
\draw (4.500,0.600) -- (4.500,11.400);
\node at (11.550,4.500) {$x$};
\node at (4.500,11.550) {$y$};
\node at (4.200,4.050) {$O$};
\draw (5.700,4.650) -- (5.700,4.350);
\node at (5.700,3.900) {$1$};
\draw (6.900,4.650) -- (6.900,4.350);
\node at (6.900,3.900) {$2$};
\draw (8.100,4.650) -- (8.100,4.350);
\node at (8.100,3.900) {$3$};
\draw (9.300,4.650) -- (9.300,4.350);
\node at (9.300,3.900) {$4$};
\draw (4.350,2.100) -- (4.650,2.100);
\node at (4.050,1.950) {$-4$};
\draw (3.000,0.600) -- (3.262,1.074) -- (3.511,1.582) -- (3.753,2.104) -- (3.994,2.621) -- (4.237,3.112) -- (4.490,3.559) -- (4.758,3.941) -- (5.045,4.238) -- (5.357,4.431) -- (5.700,4.500) -- (6.043,4.433) -- (6.353,4.250) -- (6.634,3.982) -- (6.890,3.655) -- (7.125,3.300) -- (7.342,2.945) -- (7.544,2.618) -- (7.735,2.350) -- (7.919,2.167) -- (8.100,2.100) -- (8.271,2.167) -- (8.426,2.350) -- (8.567,2.618) -- (8.695,2.945) -- (8.812,3.300) -- (8.921,3.655) -- (9.022,3.982) -- (9.118,4.250) -- (9.210,4.433) -- (9.300,4.500) -- (9.407,4.585) -- (9.542,4.824) -- (9.700,5.189) -- (9.871,5.652) -- (10.050,6.188) -- (10.229,6.768) -- (10.400,7.367) -- (10.558,7.956) -- (10.693,8.509) -- (10.800,9.000);
\draw (2.400,0.600) -- (2.752,1.074) -- (3.089,1.582) -- (3.415,2.104) -- (3.734,2.621) -- (4.050,3.112) -- (4.366,3.559) -- (4.685,3.941) -- (5.011,4.238) -- (5.348,4.431) -- (5.700,4.500) -- (6.035,4.433) -- (6.324,4.250) -- (6.577,3.982) -- (6.804,3.655) -- (7.013,3.300) -- (7.212,2.945) -- (7.411,2.618) -- (7.620,2.350) -- (7.846,2.167) -- (8.100,2.100) -- (8.352,2.167) -- (8.570,2.350) -- (8.756,2.618) -- (8.911,2.945) -- (9.037,3.300) -- (9.137,3.655) -- (9.211,3.982) -- (9.262,4.250) -- (9.291,4.433) -- (9.300,4.500) -- (9.407,4.587) -- (9.545,4.836) -- (9.708,5.229) -- (9.890,5.748) -- (10.088,6.375) -- (10.294,7.092) -- (10.503,7.881) -- (10.711,8.724) -- (10.912,9.603) -- (11.100,10.500);
\draw (5.700,4.500) circle (0.090);
\node at (5.700,4.800) {$(1, 0)$};
\draw (8.100,2.100) circle (0.090);
\node at (8.100,1.650) {$(3, -4)$};
\draw (9.300,4.500) circle (0.090);
\node at (9.450,4.800) {$(4, 0)$};
\draw (4.500,2.100) circle (0.090);
\node at (4.350,1.650) {$(0, -4)$};
\end{tikzpicture}
\end{minipage}\hfill\null

\item 造型∾可以做成美丽的丝带,将其看作图中曲线 $C$ 的一部分. 已知 $C$ 过坐标原点 $O$. 且 $C$ 上的点满足横坐标大于 $-2$,到点 $F(2,0)$ 的距离与到定直线 $x=a(a<0)$ 的距离之积为 $4$,则( )
\\
\choice{$a=-2$}{点 $(2\sqrt{2},0)$ 在 $C$ 上}{$C$ 在第一象限的点的纵坐标的最大值为 $1$}{当点 $(x_0, y_0)$ 在 $C$ 上时, $y_0 \le \frac{4}{x_0+2}$}

\par\noindent\hfill\begin{minipage}{0.48\textwidth}\centering
\begin{tikzpicture}[>=Stealth, scale=0.8, line width=0.5pt]
\draw (0.000,6.000) -- (11.700,6.000);
\draw (4.500,0.000) -- (4.500,11.700);
\node at (11.400,5.400) {$x$};
\node at (4.800,11.250) {$y$};
\node at (4.050,5.400) {$O$};
\draw (4.500,6.000) -- (4.329,6.711) -- (4.175,7.402) -- (4.037,8.071) -- (3.914,8.717) -- (3.806,9.338) -- (3.712,9.931) -- (3.630,10.496) -- (3.559,11.030) -- (3.500,11.532) -- (3.450,12.000);
\draw (4.500,6.000) -- (4.329,5.289) -- (4.175,4.598) -- (4.037,3.929) -- (3.914,3.283) -- (3.806,2.663) -- (3.712,2.069) -- (3.630,1.504) -- (3.559,0.970) -- (3.500,0.468) -- (3.450,0.000);
\draw (4.500,6.000) -- (4.802,6.648) -- (5.152,7.152) -- (5.526,7.512) -- (5.901,7.728) -- (6.251,7.800) -- (6.554,7.728) -- (6.785,7.512) -- (6.921,7.152) -- (6.937,6.648) -- (6.810,6.000) -- (6.937,5.352) -- (6.921,4.848) -- (6.785,4.488) -- (6.554,4.272) -- (6.251,4.200) -- (5.901,4.272) -- (5.526,4.488) -- (5.152,4.848) -- (4.802,5.352) -- (4.500,6.000);
\draw (5.700,6.000) circle (0.090);
\node at (5.550,5.400) {$F$};
\end{tikzpicture}
\end{minipage}\hfill\null

\end{enumerate}

\begin{enumerate}[align=left,labelindent=0em,labelwidth=2em,labelsep=0em,leftmargin=2em]
\item[{\bf 二、}]{\bf\sf 填空题:本题共 3 小题,每小题 5 分,共 15 分。}
\end{enumerate}
\begin{enumerate}[align=left,labelindent=0em,label={\bf\sf\arabic*.},labelwidth=1.5em,labelsep=0em,leftmargin=1.5em,itemsep=0pt,topsep=0pt,start=12]
\item 设双曲线 $C: \frac{x^2}{a^2} - \frac{y^2}{b^2} = 1(a > 0, b > 0)$ 的左右焦点分别为 $F_1$、$F_2$,过 $F_2$ 作平行于 $y$ 轴的直线交 $C$ 于 $A$、$B$ 两点,若 $|F_1 A| = 13$,$|AB| = 10$,则 $C$ 的离心率为\undsp .

\par\noindent\hfill\begin{minipage}{0.48\textwidth}\centering
\begin{tikzpicture}[>=Stealth, scale=0.8, line width=0.5pt]
\draw (0.000,6.000) -- (12.000,6.000);
\draw (6.000,12.000) -- (6.000,0.000);
\node at (11.550,6.150) {$x$};
\node at (6.150,11.550) {$y$};
\draw (10.500,10.050) -- (10.000,9.325) -- (9.576,8.678) -- (9.224,8.107) -- (8.940,7.609) -- (8.719,7.181) -- (8.556,6.821) -- (8.447,6.525) -- (8.388,6.292) -- (8.374,6.117) -- (8.400,6.000) -- (8.447,5.883) -- (8.503,5.708) -- (8.580,5.475) -- (8.686,5.179) -- (8.831,4.819) -- (9.026,4.391) -- (9.281,3.893) -- (9.605,3.322) -- (10.008,2.675) -- (10.500,1.950);
\draw (1.500,10.050) -- (2.000,9.325) -- (2.424,8.678) -- (2.776,8.107) -- (3.060,7.609) -- (3.281,7.181) -- (3.444,6.821) -- (3.553,6.525) -- (3.612,6.292) -- (3.626,6.117) -- (3.600,6.000) -- (3.553,5.883) -- (3.497,5.708) -- (3.420,5.475) -- (3.314,5.179) -- (3.169,4.819) -- (2.974,4.391) -- (2.719,3.893) -- (2.395,3.322) -- (1.992,2.675) -- (1.500,1.950);
\draw (2.400,6.000) circle (0.090);
\node at (2.100,5.400) {$F₁$};
\draw (9.600,6.000) circle (0.090);
\node at (9.750,5.400) {$F₂$};
\draw (9.600,9.000) circle (0.090);
\node at (9.750,9.150) {$A$};
\draw (9.600,3.000) circle (0.090);
\node at (9.750,2.850) {$B$};
\draw (9.600,9.000) -- (9.600,3.000);
\draw[dashed] (2.400,6.000) -- (9.600,9.000);
\end{tikzpicture}
\end{minipage}\hfill\null

\item 若曲线 $y = e^x + x$ 在点 $(0, 1)$ 处的切线也是曲线 $y = \ln(x+1) + a$ 的切线,则 $a = \_\_\_\_\_\_\_\_.$
\item 甲、乙两人各有四张卡片,每张卡片上标有一个数字,甲的卡片上分别标有数字 $1, 3, 5, 7$,乙的卡片上分别标有数字 $2, 4, 6, 8$,两人进行四轮比赛,在每轮比赛中,两人各自从自己持有的卡片中随机选一张,并比较所选卡片上数字的大小,数字大的人得 $1$ 分,数字小的人得 $0$ 分,然后各自弃置此轮所选的卡片(弃置的卡片在此后的轮次中不能使用)。则四轮比赛后,甲的总得分不小于 $2$ 的概率为 \undsp 。
\end{enumerate}

\begin{enumerate}[align=left,labelindent=0em,labelwidth=2em,labelsep=0em,leftmargin=2em]
\item[{\bf 三、}]{\bf\sf 解答题:本题共 4 小题,共 64 分。解答应写出文字说明、证明过程或演算步骤。}
\end{enumerate}
\begin{enumerate}[align=left,labelindent=0em,label={\bf\sf\arabic*.},labelwidth=1.5em,labelsep=0em,leftmargin=1.5em,itemsep=0pt,topsep=0pt,start=15]
\item 已知 $A(0,3)$ 和 $P\left(3,\frac{3}{2}\right)$ 为椭圆 $C:\frac{x^2}{a^2}+\frac{y^2}{b^2}=1(a>b>0)$ 上两点。

(1) 求 $C$ 的离心率;

(2) 若过 $P$ 的直线 $l$ 交 $C$ 于另一点 $B$,且 $\triangle ABP$ 的面积为9,求 $l$ 的方程。

\par\noindent\hfill\begin{minipage}{0.48\textwidth}\centering
\begin{tikzpicture}[>=Stealth, scale=0.8, line width=0.5pt]
\draw (0.300,6.000) -- (11.700,6.000);
\draw (6.000,0.300) -- (6.000,11.700);
\node at (11.550,5.550) {$x$};
\node at (6.150,11.400) {$y$};
\node at (5.700,5.550) {$O$};
\draw (6.000,6.000) ellipse (3.117 and 2.700);
\draw (6.000,8.700) circle (0.090);
\node at (6.150,8.850) {$A$};
\draw (8.700,7.350) circle (0.090);
\node at (8.850,7.500) {$P$};
\draw (3.300,4.650) circle (0.090);
\node at (2.700,4.200) {$B₁$};
\draw (6.000,3.300) circle (0.090);
\node at (6.150,2.850) {$B₂$};
\draw[dashed] (8.700,7.350) -- (3.300,4.650);
\draw[dashed] (8.700,7.350) -- (6.000,3.300);
\draw (6.000,8.700) -- (8.700,7.350) -- (3.300,4.650) -- (6.000,8.700);
\draw (6.000,8.700) -- (8.700,7.350) -- (6.000,3.300) -- (6.000,8.700);
\end{tikzpicture}
\end{minipage}\hfill\null


\vspace{10em}
\item 如图,四棱锥 $P-ABCD$ 中,$PA \perp$ 底面 $ABCD$,$PA=AC=2$,$BC=1, AB=\sqrt{3}$.

(1)若 $AD \perp PB$,证明:$AD // $ 平面 $PBC$;

(2)若 $AD \perp DC$,且二面角 $A-CP-D$ 的正弦值为 $\frac{\sqrt{42}}{7}$,求 $AD$.

\par\noindent\hfill\begin{minipage}{0.48\textwidth}\centering
\begin{tikzpicture}[>=Stealth, scale=0.8, line width=0.5pt]
\draw (3.000,9.000) -- (3.000,3.000);
\draw (3.000,9.000) -- (6.000,1.500);
\draw (3.000,9.000) -- (10.500,3.000);
\draw[dashed] (3.000,9.000) -- (6.000,3.900);
\draw (3.000,3.000) -- (6.000,1.500);
\draw (6.000,1.500) -- (10.500,3.000);
\draw[dashed] (10.500,3.000) -- (6.000,3.900);
\draw[dashed] (6.000,3.900) -- (3.000,3.000);
\draw[dashed] (3.000,3.000) -- (10.500,3.000);
\node at (2.700,9.300) {$P$};
\node at (2.550,2.550) {$A$};
\node at (5.850,1.050) {$B$};
\node at (10.800,2.850) {$C$};
\node at (5.700,4.050) {$D$};
\end{tikzpicture}
\end{minipage}\hfill\null


\vspace{10em}
\item 已知函数 $f(x) = \ln \frac{x}{2-x} + ax + b(x-1)^3$

(1) 若 $b=0$,且 $f'(x) \ge 0$,求 $a$ 的最小值;

(2) 证明:曲线 $y=f(x)$ 是中心对称图形;

(3) 若 $f(x) > -2$ 当且仅当 $1 < x < 2$,求 $b$ 的取值范围.

\vspace{10em}
\item 设 $m$ 为正整数,数列 $a_1, a_2, \dots, a_{4m+2}$ 是公差不为 $0$ 的等差数列,若从中删去两项 $a_i$ 和 $a_j(i<j)$ 后剩余的 $4m$ 项可被平均分为 $m$ 组,且每组的 $4$ 个数都能构成等差数列,则称数列 $a_1, a_2, \dots, a_{4m+2}$ 是 $(i,j)$-可分数列。

(1) 写出所有的 $(i,j)$, $1 \le i < j \le 6$,使数列 $a_1, a_2, \dots, a_6$ 是 $(i,j)$-可分数列;

(2) 当 $m \ge 3$ 时,证明:数列 $a_1, a_2, \dots, a_{4m+2}$ 是 $(2,13)$-可分数列;

(3) 从 $1, 2, \dots, 4m+2$ 中一次任取两个数 $i$ 和 $j(i<j)$,记数列 $a_1, a_2, \dots, a_{4m+2}$ 是 $(i,j)$-可分数列的概率为 $P_m$,证明:$P_m > \frac{1}{8}$。

\vspace{10em}
\end{enumerate}
\end{document}