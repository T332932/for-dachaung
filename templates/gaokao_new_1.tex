% 2025 全国卷Ⅰ 数学(骨架模板,仅结构/版式,无题目内容)
\documentclass[12pt]{ctexart}
\usepackage{amsmath,amssymb}
\usepackage{geometry,graphicx,enumitem}
\geometry{paperheight=26cm,paperwidth=18.4cm,left=2cm,right=2cm,top=1.5cm,bottom=2cm}
\setlength{\parskip}{0.6em}
\setlength{\parindent}{0pt}

\usepackage{fancyhdr}
\pagestyle{fancy}
\renewcommand{\headrulewidth}{0pt}
\fancyfoot[C]{\bf 数学试题A 第 \thepage 页}

% 简单圈号,用粗体数字代替
\newcommand{\quan}[1]{\textbf{#1}}

% 题目项命令:\qitem{分值}{题干内容}
\newcommand{\qitem}[2]{\item (#1分)\; #2}

\begin{document}
% 头部信息
\noindent{\heiti 绝密$\star$启用前\hfill 试卷类型:{\bf A}}
\begin{center}
  \zihao{3} 2025年普通高等院校招生全国统一考试\\
  \zihao{2}\heiti 数学
\end{center}

{\bf 本试卷共5页,19小题,满分150分,考试用时120分钟。}

% 注意事项
{\zihao{5}
\begin{enumerate}[align=left,labelindent=0em,labelwidth=4.5em,leftmargin=5em]
  \item[{\heiti 注意事项:}]
  \begin{enumerate}[label=\arabic*.,leftmargin=1em]
    \item 答题前,考生先将自己的姓名、准考证号填写在答题卡上。
    \item 选择题用 2B 铅笔填涂,非选择题用黑色字迹钢笔或签字笔作答。
    \item 答案必须写在答题卡指定区域,保持整洁。
    \item 考试结束后,将试卷和答题卡一并交回。
  \end{enumerate}
\end{enumerate}
}
\vspace{1em}

% 一、单选题 8 题 × 5 分
\begin{enumerate}[align=left,labelindent=0em,labelwidth=2em,labelsep=0em,leftmargin=2em]
  \item[{\bf 一、}]{\bf 选择题:本大题共 8 小题,每小题 5 分,共 40 分.每小题给出的四个选项中,只有一个选项是正确的.}
\end{enumerate}
\begin{enumerate}[align=left,labelindent=0em,label={\bf\arabic*.},labelwidth=1.5em,labelsep=0em,leftmargin=1.5em]
  %% 单选 1–8
  \qitem{5}{\textit{单选题1题干占位……}}
  \qitem{5}{\textit{单选题2题干占位……}}
  \qitem{5}{\textit{单选题3题干占位……}}
  \qitem{5}{\textit{单选题4题干占位……}}
  \qitem{5}{\textit{单选题5题干占位……}}
  \qitem{5}{\textit{单选题6题干占位……}}
  \qitem{5}{\textit{单选题7题干占位……}}
  \qitem{5}{\textit{单选题8题干占位……}}
\end{enumerate}

% 二、多选题 3 题 × 6 分
\begin{enumerate}[align=left,labelindent=0em,labelwidth=2em,labelsep=0em,leftmargin=2em,start=9]
  \item[{\bf 二、}]{\bf 选择题:本题共 3 小题,每小题 6 分,共 18 分。在每小题给出的选项中,有多项符合题目要求。全部选对得 6 分,部分选对得部分分,有选错得 0 分。}
\end{enumerate}
\begin{enumerate}[align=left,labelindent=0em,label={\bf\arabic*.},labelwidth=1.5em,labelsep=0em,leftmargin=1.5em,start=9]
  %% 多选 9–11
  \qitem{6}{\textit{多选题9题干占位……}}
  \qitem{6}{\textit{多选题10题干占位……}}
  \qitem{6}{\textit{多选题11题干占位……}}
\end{enumerate}

% 三、填空题 3 题 × 5 分
\begin{enumerate}[align=left,labelindent=0em,labelwidth=2em,labelsep=0em,leftmargin=2em,start=12]
  \item[{\bf 三、}]{\bf 填空题:本大题共 3 小题,每小题 5 分,共 15 分。}
\end{enumerate}
\begin{enumerate}[align=left,labelindent=0em,label={\bf\arabic*.},labelwidth=1.5em,labelsep=0em,leftmargin=1.5em,resume]
  %% 填空 12–14
  \qitem{5}{\textit{填空题12题干占位……} \underline{\hspace{3em}}}
  \qitem{5}{\textit{填空题13题干占位……} \underline{\hspace{3em}}}
  \qitem{5}{\textit{填空题14题干占位……} \underline{\hspace{3em}}}
\end{enumerate}

% 四、解答题 5 题(13,15,15,17,17)
\begin{enumerate}[align=left,labelindent=0em,labelwidth=2em,labelsep=0em,leftmargin=2em,start=15]
  \item[{\bf 四、}]{\bf 解答题:本大题共 5 小题,共 77 分。解答应写出文字说明、证明过程或演算步骤。}
\end{enumerate}
\begin{enumerate}[align=left,labelindent=0em,label={\bf\arabic*.},labelwidth=1.5em,labelsep=0em,leftmargin=1.5em,start=15]
  %% 解答 15–19
  \qitem{13}{\textit{解答题15题干占位……}}
  \vspace{10em}
  \qitem{15}{\textit{解答题16题干占位……}}
  \vspace{10em}
  \qitem{15}{\textit{解答题17题干占位……}}
  \vspace{10em}
  \qitem{17}{\textit{解答题18题干占位……}}
  \vspace{10em}
  \qitem{17}{\textit{解答题19题干占位……}}
  \vspace{10em}
\end{enumerate}

% 尾注(可选)
\begin{center}
  {\heiti 严禁用于商业用途,转载请注明作者与出处!}
\end{center}

\end{document}
